\documentclass[12pt]{article}
\usepackage{amsfonts}
\usepackage{amstext}
\usepackage{amssymb}
\usepackage{txfonts}
\usepackage{pifont}
\usepackage{bbm}
\usepackage{mathrsfs}
\usepackage{graphicx}
\usepackage{fancyhdr}
\usepackage{lastpage}
\lhead{Team 1234} \chead{} \rhead{Page \thepage{} of
\pageref{LastPage}} \lfoot{} \cfoot{} \rfoot{} \pagestyle{fancy}

\author{Team 1234}
\title{Applying Voronoi Diagrams to the Redistricting Problem}

\begin{document}
\maketitle
\begin{center}
\textbf{Abstract}
\end{center}
Gerrymandering is an issue plaguing legislative redistricting
resulting from inadequate regulation. Here, we present a novel
approach to the redistricting problem, an approach that uses a
state��s population distribution to draw the legislative boundaries.
Our method utilizes \emph{Voronoi} and population-weighted
\emph{Voronoiesque} diagrams, and was chosen for the simplicity of
the generated partition: \emph{Voronoi} regions are contiguous,
compact, and simple to generate. We found regions drawn with
Voronoiesque diagrams attained small population variance and
relative geometric simplicity. As a concrete example, we applied our
methods to partition New York state. Since New York must be divided
into 29 legislative districts, each receives roughly 3.44 \% of the
population. Our Voronoiesque diagram method generated 29 regions
with an average population of (3.34 $\pm$ 0.74)\%. We discuss
several refinements that could be made to the methods presented
which may result in smaller population variation between regions
while maintaining the simplicity of the regions and objectivity of
the method. Finally, we provide a short statement that could be
issued to the voters of New York state to explain our method and
justify its fairness to them.

\newpage
\tableofcontents \listoffigures
\newpage

\section{Introduction}
Defining Congressional districts has long been a source of
controversy in the United States. Since the district-drawers are
chosen by those currently in power, the boundaries are often created
to influence future elections by grouping an unfavorable minority
demographic with a favorable majority; this process is called
\emph{Gerrymandering}. It is common for districts to take on bizarre
shapes, spanning slim sections of multiple cities and criss-crossing
the countryside in a haphazard fashion. The only lawful restrictions
on legislative boundaries stipulate that districts must contain
equal populations, but the makeup of the districts is left entirely
to the district-drawers.

In the United Kingdom and Canada, the districts are more compact and
intuitive. Their success in mitigating Gerrymandering is attributed
to having turned over the task of boundary-drawing to nonpartisan
advisory panels. However, these independent com- missions can take
2--3 years to finalize a new division plan, calling their
effectiveness into question. It seems clear that the U.S. should
establish similar unbiased commissions, yet make some effort to
increase the efficiency of these groups. Accordingly, our goal is to
develop a small toolbox that aids in the redistricting process.
Specifically, we will create a model that draws legislative
boundaries using simple geometric constructions.

\subsection{Current Models}
The majority of methods for creating districts fall into two
categories: ones that depend on a current division arrangement (most
commonly counties) and ones that do not depend on current divisions.
Most fall into the former category. By using current divisions, the
problem is reduced to grouping these divisions in a desirable way
using a multitude of mathematical procedures. Mehrotra et.al. uses
graph partitioning theory to cluster counties to total population
variation of around 2\% of the average district size~\cite{8}. Hess
and Weaver use an iterative process to define population centroids,
use integer programming to group counties into equally populated
districts, and then reiterate the process until the centroids reach
a limit~\cite{5}. Grafinkel and Nemhauser use iterative matrix
operations to search for district combinations that are contiguous
and compact~\cite{3}. Kaiser begins with the current districts and
systematically swaps populations with adjacent districts~\cite{4}.
All of these methods use counties as their divisions since they
partition the state into a relatively small number of sections. This
is necessary because most of the mathematical tools they use become
slow and imprecise with many divisions. (This is the same as saying
they become unusable in the limit when the state is divided into
more continuous sections.) Thus using small divisions, like zip
codes which on average are 5 times smaller than a county in New
York, becomes impractical.

The other category of methods is less common. Out of all our
researched papers and documentation, there were only two methods
that did not depend on current state divisions. Forrest��s method
continually divides a state into halves while maintaining population
equality until the required number of districts is
satisfied~\cite{4,5}. Hale, Ransom and Ramsey create pie-shaped
wedges about population centers. This creates homogeneous districts
which all contain portions of a large city, suburbs, and less
populated areas~\cite{4}. These approaches are noted for being the
least biased since their only consideration is population equality
and do not use preexisting divisions. Also, they are straightforward
to apply. However, they do not consider any other possibly important
considerations for districts, such as: geographic freaures of the
state or how well they encompass cities.

\subsection{Developing Our Approach}
\label{sec:doa}
Since our goal is to create new methods that add to
the diversity of models available to a committee, we should focus on
creating district boundaries independently of current divisions. Not
only has this approach not been explored to its fullest, but it is
not obvious why counties are a good beginning point for a model:
Counties are created in the same arbitrary way as districts, so they
might also contain biases, since counties are typically not much
smaller than districts. Many of the division dependent models end up
relaxing their boundaries from county lines in order to maintain
equal populations, which makes the initial assumption of using
county divisions useless, and also allows for gerrymandering if this
relaxation method is not well regulated.

Treating the state as continuous (i.e. without preexisting
divisions) does not lead to any specific type of approach. It gives
us a lot of freedom, but at the same time we can impose more
conditions. If the Forrest and Hale et.al. methods are any
indication, we should focus on keeping cities within districts and
introduce geographical considerations. (Note that these conditions
do not have to be considered if we were to treat the problem
discretely because current divisions, like counties, are probably
dependent on prominent geographical features.) \textbf{
\begin{description}
\item{Goal:} Create a method for redistricting a state by treating the state continuously.
We require the final districts to contain equal populations and be
contiguous. Additionally, the districts should be as simple as
possible (see \ding {72}\ref{sec:notation} for a definition of
simple) and optimally take into account important geographical
features of the state.
\end{description}}

\section{Notations and Definitions}
\label{sec:notation}
\begin{itemize}
\item{\textbf{contiguous:}} A set $R$ is contiguous if it is
pathwise-connected.
\item{\textbf{compactness:}} We would like the definition of compactness to be intuitive. One
way to look at compactness is the ratio of the area of a bounded
region to the square of its perimeter. In other words
\begin{displaymath}
C_R = \frac{A_R}{{p^2}_R} = \frac{1}{4\pi}\varrhoup
\end{displaymath}
where $C_R$ is the compactness of region $R$, $A_R$ is the area,
$p_R$ is the perimeter and $\varrhoup$ is the isoperimetric
quotient. We do not explicitely use this equation, but we do keep
this idea in mind when we evaluate our model.
\item{\textbf{simple:}} Simple regions are compact and convex. Note that this describes a relative
quality, so we can compare regions by their simplicity.
\item{\textbf{Voronoi diagrams:}} a partition of the plane with respect to n nodes in the plane
such that points in the plane are in the same region of a node if
they are closer to that node than to any other point (for a detailed
description, see \ding {72}\ref{sec:vd})
\item{\textbf{generator point:}} a node of a Voronoi diagram
\item{\textbf{degeneracy:}} the number of districts represented by
on generator point
\item{\textbf{Voronoiesque diagram:}} a variation of the Voronoi
diagram based on equal masses of the regions (see \ding
{72}\ref{sec:vd})
\item{\textbf{population center:}} a region of high population
density
\end{itemize}

\section{Theoretical Evaluation of our Model}
How we analyze our model's results is a tricky affair since there is
disagreement in the redistricting literature on key issues.
\textbf{Population equality} is the most well defined. By law, the
populations within districts have to be the same to within a few
percent of the average population per district. No specific
percentage is given, but be assumed to be around 5\%.

Creating a successful redistricting model also requires
\textbf{contiguity}. In accordance with state law, districts need to
be path-wise connected so that one part of a district cannot be on
one side of the state and the other part on the other end of the
state. This requirement is meant to maintain locality and community
within districts. It does not, however, restrict islands districts
from including islands if the island��s population is below the
required population level.

Finally, there is a desire for the districts to be, in one word,
simple. There is little to no agreement on this characteristic, and
the most common terminology for this is compact. Taylor de?nes
simple as a measure of divergence from compactness due to
indentation of the boundary and gives an equation relating the
non-reflexive and reflexive interior angles of a region��s
boundary~\cite{9}. Young provides seven more measures of
compactness. The \emph{Roeck} test is a ratio of the area of the
largest inscribable circle in a region to the area of that region.
The \emph{Schwartzberg} test takes ratio between the adjusted
perimeter of a region to a the perimeter of a circle whose area is
the same as the area of the region. The \emph{moment of inertia}
test measures relative compactness by comparing ``moments of
inertia'' of different district arrangements. The \emph{Boyce-Clark}
test compares the difference between points on a district's boundary
and the center of mass of that district, where zero deviation of
these differences is most desirable. The \emph{perimeter} test
compares different district arrangements buy computing the total
perimeter of each. Finally, there is the \emph{visual} test. This
test decides simplicity based on intuition~\cite{11}.

Young notes that ``a measure [of compactness] only indicates when a
plan is more compact than another''~\cite{11}. Thus, \emph{not only
is there no consensus on how to analyze redistricting proposals, it
is also difficult to compare them.}

Finally, we remark that the above list only constrains the shape of
generated districts. We have not mentioned of any other potentially
relevant feature. For instance, it does not consider how well
populations are distributed or how well the new district boundaries
conform with other boundaries, like counties or zip codes. Even with
this short list, it is clear that we are not in a position to de?ne
a rigorous definition of simplicity. What we can do, however, is
identify features of our proposed districts which are simple and
which are not. This is in line with our goal defined in
sec.~\ref{sec:doa}, since this list can be provided to a districting
commission who decide how relevant those simple features are.
\textbf{We do not explicitly define \emph{simple}, we loosely
evaluate simplicity based on overall contiguity, compactness,
convexity, and intuitiveness of the model's districts.}

\section{Method Description}
Our approach depends heavily on using Voronoi diagrams. We begin
with a definition, its features, and motivate its application to
redistricting.

\subsection{Voronoi Diagrams}
\label{sec:vd} A Voronoi diagram is a set of polygons, called
Voronoi polygons, formed with respect to $n$ generator points
contained in the plane. Each generator $p_i$ is contained with a
Voronoi polygon $V(p_i)$ with the following property:
\begin{displaymath}
V(p_i) = \{q|d(p_i,q) \leq d(p_j,q), i\neq j\}\;\textrm{where
$d(x,y)$ is the distance from point $x$ to $y$}
\end{displaymath}

That is, the set of all such $q$ is the set of points closer to
$p_i$ than to any other $p_j$. Then the diagram is given by (see
fig~\ref{fig:1})
\begin{displaymath}
\textbf{V} = \{V(p_1), \ldots , V(p_n)\}
\end{displaymath}

Note that there is no assumption on the metric we use. Out of the
many possible choices, we use the three most common:
\begin{itemize}
\item{Euclidean Metric:} $d(p,q) = \sqrt{(x_p-x_q)^2+(y_p-y_q)^2}$
\item{Manhattan Metric:} $d(p,q) = |x_p-x_q| + |y_p-y_q|$
\item{Uniform Metric:} $d(p,q) = max\{|x_p-x_q| , |y_p-y_q|\}$
\end{itemize}

\subsubsection{Useful Features of Voronoi Diagrams}
Here is a summary of relevant properties:
\begin{itemize}
\item The Voronoi diagram for a given set of generator points is unique and produces
polygons, which are path connected.
\item The nearest generator point to $p_i$ determines an edge of
$V(p_i)$
\item The polygonal lines of a Voronoi polygon do not intersect the generator points.
\item When working in the Euclidean metric, all regions are convex.
\end{itemize}

These properties are important for our model. The first property
tells us that regardless of how we choose our generator points we
generate unique regions. This is good when considering the politics
of Gerrymandering. The second property implies that each region is
defined in terms of the surrounding generator points while in turn,
each region is relatively compact. \textbf{These features of Voronoi
diagrams effectively satisfy two out of the three criteria for
partitioning a region: contiguity and simplicity.}

\subsection{Voronoiesque Diagrams}
The second method we use to create regions is a modification of the
intuitive construction of Voronoi diagrams. The method does not fall
under the definition of Voronoi diagrams, but since it is similar to
Voronoi diagrams, we call them Voronoiesque diagrams. One way to
visualize the construction of Voronoi diagrams is to imagine shapes
(determined by the metric) that grow radially outward at a constant
rate from each generator point. In the Euclidean metric these shapes
are circles. In the Manhattan metric they are diamonds. In the
Uniform metric, they are squares. The interior of these shapes form
the regions of the diagram. As the regions intersect, they form the
boundary lines for the regions. With this picture in mind, we define
Voronoiesque diagrams to be the boundaries defined by the
intersections of these growing shapes. The fundamental difference
between Voronoi and Voronoiesque diagrams is that Voronoiesque
diagrams grow the shapes radially outward at a constant rate like
Voronoi diagrams. Their radial growth is defined with respect to
some real function on a subset of $\mathbb{R}^2$ (representing the
space on which the diagram is being generated). See fig.\ref{fig:2}

More rigorously, we define a Voronoi diagram to be the intersections
of the $\nu_i^{(t)}$'s
\subsection{Determining Generator Points Using Population Density
Distributions}
\subsection{Procudure for Creating Regions using Voronoi and
Voronoiesque Diagrams}

\section{Redistricting in New York State}
\subsection{Population Density Map}
\subsection{Limitations of Image-Based Density Map}
\subsection{Selecting Generator Points}
\subsection{Applying Voronoi Diagrams to NY}
\subsection{Applying Voronoiesque Diagrams to NY}
\subsection{Precisely Defining Boundary Lines}

\section{Analysis}
\subsection{New York State Results}
\subsection{General Results}

\section{Improving the Method}
\subsection{Boundary Refinement}
\subsection{Geographic Obstacles}

\section{Bulletin to the Voters of the State of New York}

\section{Conclusion}

\begin{thebibliography}{99}
\bibitem{1} U.S. Census Bureau. 2005 forst edition tiger/line data, Feb. 2007.
\end{thebibliography}

\end{document}
